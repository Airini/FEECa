% This template is losely based on the abstract template from Conference
% on Geometry: Theory and Applications (CGTA 2015), and uses code
% listings style ``anslistings'' from the Archive of Numerical Software.
\documentclass[11pt,a4paper]{article}
%\usepackage[square]{natbib}
\usepackage{amsmath}
\usepackage{amsfonts}
\usepackage{amssymb}
\usepackage{graphicx}
\usepackage{algorithm}
\usepackage{algorithmicx}
\usepackage{tipa}
\usepackage{algpseudocode}
%\usepackage{anslistings} % <-- should be provided with this template

\pagestyle{empty}
\usepackage[left=25mm, right=25mm, top=15mm, bottom=20mm, noheadfoot]{geometry}
% please don't change geometry settings!

%\usepackage[T1]{fontenc}   %% get hyphenation and accented letters right
%\usepackage{mathptmx}      %% use fitting times fonts also in formulas

%%%

\begin{document}
\thispagestyle{empty}

% Insert title here:
\title{FEECa: Finite Element Exterior Calculus in Haskell}

% Insert authors and afiliations here, speaker in bold:
\author{Irene Lobo Valbuena, Chalmers University of Technology, lobo@chalmers.se \\
        \textbf{Simon Pfreundschuh}, Chalmers University of Technology, simonpf@student.chalmers.se}

\date{} % please leave date empty
\maketitle\thispagestyle{empty}

% Insert keywords here
Keywords: \emph{Finite Element Exterior Calculus, Haskell, Functional Programming,
  Basis Functions}\\

% Insert text here:

FEECa (\textipa{["fi:ka]}) is a framework for finite element exterior
calculus written in Haskell. Starting out as a project to explore the
possibilities of using a functional programming language to formulate
abstract mathematical concepts, the project has developed into a
full-fledged generator of basis form generator and framework for
computations on differential forms. FEECa supports differential forms
in arbitrary dimensions and implements monomial as well as Bernstein bases
for polynomials. Furthermore, the framwork  provides functionality to
compute bases of finite element spaces based on the geometric composition
proposed by Arnold et al.\cite{arnold}.

\begin{thebibliography}{00}
\addcontentsline{toc}{chapter}{References}
\bibitem{arnold} Arnold, Douglas N. and Falk, Richard S. and Winther, Ragnar:
 Geometric decompositions and local bases for spaces of finite element differential forms
\end{thebibliography}

\end{document}
