% This template is losely based on the abstract template from Conference
% on Geometry: Theory and Applications (CGTA 2015), and uses code
% listings style ``anslistings'' from the Archive of Numerical Software.
\documentclass[11pt,a4paper]{article}
%\usepackage[square]{natbib}
\usepackage{amsmath}
\usepackage{amsfonts}
\usepackage{amssymb}
\usepackage{graphicx}
\usepackage{algorithm}
\usepackage{algorithmicx}
\usepackage{tipa}
\usepackage{algpseudocode}
%\usepackage{anslistings} % <-- should be provided with this template

\pagestyle{empty}
\usepackage[left=25mm, right=25mm, top=15mm, bottom=20mm, noheadfoot]{geometry}
% please don't change geometry settings!

%\usepackage[T1]{fontenc}   %% get hyphenation and accented letters right
%\usepackage{mathptmx}      %% use fitting times fonts also in formulas

%%%

\begin{document}
\thispagestyle{empty}

% Insert title here:
\title{FEECa: Finite Element Exterior Calculus in Haskell}

% Insert authors and afiliations here, speaker in bold:
\author{Irene Lobo Valbuena, Chalmers University of Technology, lobo@chalmers.se \\
        \textbf{Simon Pfreundschuh}, Chalmers University of Technology, simonpf@student.chalmers.se}

\date{} % please leave date empty
\maketitle\thispagestyle{empty}

% Insert keywords here
Keywords: \emph{Finite Element Exterior Calculus, Haskell, Functional Programming,
  Basis Functions}\\

% Insert text here:

The theory of finite element exterior calculus (FEEC), developed by
Arnold, Falk and Winther \cite{arnold1}, is a mathematical framework
for the discretization of partial differential equations (PDEs) that
allows for a universal treatment of a large number of physical
problems.  FEECa (\textipa{["fi:ka]}) is a software package written in
Haskell that implements this mathematical framework. Starting out as a
project to explore the possibilities of using a functional programming
language to implement the abstract, mathematical concepts of finite
element exterior calculus, the project developed into a full-fledged
basis form generator and framework for computations on differential
forms. FEECa handles polynomial differential forms in arbitrary
dimensions and implements monomial as well as Bernstein bases for
polynomials. The package provides functionality to compute bases of
the $\mathcal{P}_r\Lambda^k$ and $\mathcal{P}^-_r\Lambda^k$ spaces of
finite elements based on the geometric composition proposed by Arnold,
 Falk and Winther in \cite{arnold2}.

\begin{thebibliography}{00}
\addcontentsline{toc}{chapter}{References}
\bibitem{arnold1} Arnold, Douglas N. and Falk, Richard S. and Winther, Ragnar.
  \textit{Finite element exterior calculus, homological techniques, and applications}.
  J. Acta Numerica. 2006
\bibitem{arnold2} Arnold, Douglas N. and Falk, Richard S. and Winther, Ragnar:
  \textit{Geometric decompositions and local bases for spaces of finite element differential forms}.
  J. Computer Methods in Applied Mechanics and Engineering. 2009.
\end{thebibliography}

\end{document}
